\documentclass{article}
\usepackage{amsmath, amsthm, amssymb, amsfonts}
\usepackage{hyperref}
\usepackage[utf8]{inputenc}
\usepackage{textgreek}
\usepackage{stmaryrd}

\title{\textbf{Measure Theory}\\[0.5em]}
\author{}
\date{}

\theoremstyle{definition}
\newtheorem{definition}{Definition}[section]
\newtheorem{theorem}{Theorem}[section]
\newtheorem{corollary}{Corollary}[theorem]
\newtheorem{lemma}[theorem]{Lemma}
\newtheorem{example}{Example}[section]
\newtheorem{remark}{Remark}[section]


\begin{document}
\maketitle

\section{σ-Algebras}
\begin{definition}
A $\sigma$-algebra $\mathcal{A}$ on a set $X$ is a family of subsets of $X$ such that:
\begin{enumerate}
    \item $X \in \mathcal{A} \hfill \text{($\boldsymbol{\Sigma}_1$)}$
    \item If $A \in \mathcal{A}$, then $A^c \in \mathcal{A} \hfill \text{($\boldsymbol{\Sigma}_2$)}$
    \item If $(A_n)_{n \in \mathbb{N}} \subseteq \mathcal{A}$, then $\bigcup_{n \in \mathbb{N}} A_n \in \mathcal{A} \hfill \text{($\boldsymbol{\Sigma}_3$)}$
\end{enumerate}

\medskip

A set $A \in \mathcal{A}$ is said to be \textit{measurable} or \textit{$\mathcal{A}$-measurable}.
\end{definition}


\medskip
\begin{example}\leavevmode\
\begin{enumerate}
    \item $\mathcal{P}(X)$ is a $\sigma$-algebra (the maximal $\sigma$-algebra on $X$).
    
    \item $\{\emptyset, X\}$ is a $\sigma$-algebra (the minimal $\sigma$-algebra on $X$).
    
    \item $A := \{A \subseteq X : \#A < \infty \text{ or } \#A^c < \infty\}$ is a $\sigma$-algebra.

    \item (Trace $\sigma$-algebra) Let $E \subseteq X$ be any set and let $\mathcal{A}$ be a $\sigma$-algebra on $X$. Then
    \[
    \mathcal{A}_E := \{E \cap A : A \in \mathcal{A}\}
    \]
    is a $\sigma$-algebra on $E$.

   \begin{proof}
We verify the three defining properties of a $\sigma$-algebra on $E$:

\begin{enumerate}
    \item Since $X \in \mathcal{A}$, we have \( E = E \cap X \in \mathcal{A}_E \).
    \item If \( E \cap A \in \mathcal{A}_E \), then \( E \setminus (E \cap A) = E \cap A^c \), and since \( A^c \in \mathcal{A} \), it follows that \( E \cap A^c \in \mathcal{A}_E \).
    \item If \( (E \cap A_n)_{n \in \mathbb{N}} \subseteq \mathcal{A}_E \), then \( \bigcup_{n} (E \cap A_n) = E \cap \bigcup_{n} A_n \), and since \( \bigcup_n A_n \in \mathcal{A} \), we conclude that \( \bigcup_n (E \cap A_n) \in \mathcal{A}_E \).
\end{enumerate}
Hence, $\mathcal{A}_E$ is a $\sigma$-algebra on $E$.
\end{proof}

    \item (Pre-image $\sigma$-algebra) Let $f : X \to X'$ be a function and let $\mathcal{A}'$ be a $\sigma$-algebra on $X'$. Then
    \[
    \mathcal{A} := \{f^{-1}(A') : A' \in \mathcal{A}'\}
    \]
    is a $\sigma$-algebra on $X$.
\end{enumerate}
\end{example}


\medskip
\begin{theorem}
Let $X$ be a set and let $\{\mathcal{A}_i : i \in I\}$ be a family of $\sigma$-algebras on $X$. Define
\[
\mathcal{A} := \bigcap_{i \in I} \mathcal{A}_i = \{ A \subseteq X : A \in \mathcal{A}_i \text{ for all } i \in I \}.
\]
Then, $\mathcal{A}$ is a $\sigma$-algebra on $X$.
\end{theorem}

\begin{proof}
We verify the $\sigma$-algebra properties for $\mathcal{A}$:
\begin{enumerate}
    \item Since $X \in \mathcal{A}_i$ for all $i \in I$, we have $X \in \mathcal{A}$.
    \item If $A \in \mathcal{A}$, then $A \in \mathcal{A}_i$ for all $i \in I$, so \( A^c = X \setminus A \in \mathcal{A}_i \) for all \( i \in I \), hence \( A^c \in \mathcal{A} \).
    \item If \( (A_n)_{n \in \mathbb{N}} \subseteq \mathcal{A} \), then \( A_n \in \mathcal{A}_i \) for all \( n \) and \( i \), so \( \bigcup_{n \in \mathbb{N}} A_n \in \mathcal{A}_i \) for all \( i \in I \), and thus \( \bigcup_{n \in \mathbb{N}} A_n \in \mathcal{A} \).
\end{enumerate}
Therefore, $\mathcal{A}$ is a $\sigma$-algebra on $X$.
\end{proof}


\medskip
\begin{definition}
Let $X$ be a set and let $\mathcal{E} \subseteq \mathcal{P}(X)$ be a collection of subsets of $X$. The \textit{$\sigma$-algebra generated by} $\mathcal{E}$, denoted by $\sigma(\mathcal{E})$, is the smallest $\sigma$-algebra on $X$ containing all sets in $\mathcal{E}$. That is,

\[
\sigma(\mathcal{E}) := \bigcap \big\{ \mathcal{A} \subseteq \mathcal{P}(X) :\, \mathcal{A} \text{ is a } \sigma\text{-algebra on } X, \; \mathcal{E} \subseteq \mathcal{A} \big\}.
\]
\end{definition}


\medskip
\begin{remark}[Generated $\sigma$-algebras]
\leavevmode
\begin{enumerate}
    \item If $\mathcal{A}$ is a $\sigma$-algebra, then $\sigma(\mathcal{A}) = \mathcal{A}$.
    \item For $A \subseteq X$, we have $\sigma(\{A\}) = \{\emptyset, A, A^c, X\}$.
    \item If $\mathcal{F} \subseteq \mathcal{G} \subseteq \mathcal{A}$, then $\sigma(\mathcal{F}) \subseteq \sigma(\mathcal{G}) \subseteq \sigma(\mathcal{A})$.
\end{enumerate}
\end{remark}

\medskip
\begin{definition}[Topological Space]
A \textit{topological space} is a pair $(X, \mathcal{T})$ where $X$ is a set and $\mathcal{T} \subseteq \mathcal{P}(X)$ is a collection of subsets of $X$, called \textit{open sets}, satisfying the following properties:
\begin{enumerate}
    \item $\emptyset \in \mathcal{T}$ and $X \in \mathcal{T}$,
    \item If $\{U_\alpha \in \mathcal{T} : \alpha \in I\}$ is an arbitrary collection of open sets, then the union $\bigcup_{\alpha \in I} U_\alpha \in \mathcal{T}$,
    \item If $\{U_i \in \mathcal{T} : i = 1, \dots, n\}$ is a finite collection of open sets, then the intersection $\bigcap_{i=1}^n U_i \in \mathcal{T}$.
\end{enumerate}
The collection $\mathcal{T}$ is called a \textit{topology} on $X$. The complement of an open set is called a \textit{closed set}.
\end{definition}


\medskip
\begin{remark}[Standard Topology on \(\mathbb{R}^n\)]
A subset \( U \subseteq \mathbb{R}^n \) is called \textit{open} if for every point \( x \in U \), there exists an \(\varepsilon > 0\) such that the open ball
\[
B_\varepsilon(x) := \{ y \in \mathbb{R}^n : \|x - y\| < \varepsilon \},
\]
where \(\|\cdot\|\) denotes the Euclidean norm, is contained in \( U \); that is, \( B_\varepsilon(x) \subseteq U \).

The collection of all such open sets is denoted by \(\mathcal{O} = \mathcal{O}_{\mathbb{R}^n}\) and forms the \textit{standard topology} on \(\mathbb{R}^n\).
\end{remark}


\medskip
\begin{definition}[Borel $\sigma$-algebra]
The \(\sigma\)-algebra \(\sigma(\mathcal{O})\) generated by the collection of open sets \(\mathcal{O} = \mathcal{O}_{\mathbb{R}^n}\) of \(\mathbb{R}^n\) is called the \textit{Borel \(\sigma\)-algebra} on \(\mathbb{R}^n\). 

Its elements are called \textit{Borel sets} or \textit{Borel measurable sets}. We denote the Borel \(\sigma\)-algebra by \(\mathcal{B}(\mathbb{R}^n)\).
\end{definition}

\medskip
\begin{definition}
Let \( X \) be a topological space and let \( A \subseteq X \). A collection \( \{U_\alpha\}_{\alpha \in A} \subseteq \mathcal{T} \) of open sets is called an \emph{open cover} of \( A \) if
\[
A \subseteq \bigcup_{\alpha \in A} U_\alpha.
\]
A \emph{subcover} is a subcollection that still covers \( A \). The set \( A \) is called \emph{compact} if every open cover of \( A \) admits a finite subcover.
\end{definition}

\medskip
\begin{remark}
In \( \mathbb{R}^n \), a set is compact if and only if it is closed and bounded (Heine–Borel Theorem).
\end{remark}

\medskip
\begin{theorem}[Borel $\sigma$-algebra from Different Generators]\label{thm:borel-generators}
Let \( \mathcal{O}, \mathcal{C}, \mathcal{K} \subseteq \mathcal{P}(\mathbb{R}^n) \) denote the collections of open, closed, and compact subsets of \( \mathbb{R}^n \), respectively. Then,
\[
\mathcal{B}(\mathbb{R}^n) = \sigma(\mathcal{O}) = \sigma(\mathcal{C}) = \sigma(\mathcal{K}).
\]
\end{theorem}

\begin{proof}
Since compact sets are closed, we have \( \mathcal{K} \subseteq \mathcal{C} \), and by Remark~1.1(3), \( \sigma(\mathcal{K}) \subseteq \sigma(\mathcal{C}) \). Conversely, for any \( C \in \mathcal{C} \), define \( C_k := C \cap B_k(0) \), where \( B_k(0) \) is the closed ball of radius \( k \) centered at the origin. Each \( C_k \) is closed and bounded, hence compact, so \( C_k \in \mathcal{K} \). Since \( C = \bigcup_{k \in \mathbb{N}} C_k \), it follows that \( C \in \sigma(\mathcal{K}) \), and thus \( \sigma(\mathcal{C}) \subseteq \sigma(\mathcal{K}) \).

Next, since \( \mathcal{C} = \mathcal{O}^c := \{ U^c : U \in \mathcal{O} \} \), and complements of sets in a σ-algebra are again in the σ-algebra, it follows that \( \mathcal{C} \subseteq \sigma(\mathcal{O}) \), hence \( \sigma(\mathcal{C}) \subseteq \sigma(\mathcal{O}) \). The reverse inclusion follows similarly from \( \mathcal{O} = \mathcal{C}^c \). Therefore, we have:
\[
\sigma(\mathcal{K}) = \sigma(\mathcal{C}) = \sigma(\mathcal{O}).
\]
\end{proof}

\paragraph{Generating Sets of the Borel Algebra.}
The Borel $\sigma$-algebra $\mathcal{B}(\mathbb{R}^n)$ can be generated by various systems of sets. Of particular importance are:

\begin{itemize}
    \item The family of open rectangles:
    \[
    \mathcal{J}_{o,n} := \left\{ (a_1, b_1) \times \cdots \times (a_n, b_n) : a_i, b_i \in \mathbb{R} \right\},
    \]
    \item The family of half-open rectangles:
    \[
    \mathcal{J}_n := \left\{ [a_1, b_1) \times \cdots \times [a_n, b_n) : a_i, b_i \in \mathbb{R} \right\}.
    \]
\end{itemize}

We denote by $\mathcal{J}_n^{\mathrm{rat}}, \mathcal{J}_{o,n}^{\mathrm{rat}}$ the subsets with rational endpoints. These sets represent intervals in $\mathbb{R}$, rectangles in $\mathbb{R}^2$, cuboids in $\mathbb{R}^3$, and hypercubes in higher dimensions.

\begin{theorem}
We have the following equality of Borel $\sigma$-algebras on $\mathbb{R}^n$:
\[
\mathcal{B}(\mathbb{R}^n) = \sigma(\mathcal{J}_n^{\mathrm{rat}}) = \sigma(\mathcal{J}_{o,n}^{\mathrm{rat}}) = \sigma(\mathcal{J}_n) = \sigma(\mathcal{J}_{o,n}),
\]
\end{theorem}


\medskip
\begin{remark}
Let \(D \subseteq \mathbb{R}\) be a dense subset, for example \(D = \mathbb{Q}\) or \(D = \mathbb{R}\). 
Then the Borel sets on \(\mathbb{R}\) can also be generated by any of the following families of intervals:
\[
\{(-\infty, a) : a \in D\}, \quad \{(-\infty, a] : a \in D\}, \quad \{(a, \infty) : a \in D\}, \quad \{[a, \infty) : a \in D\}.
\]
\end{remark}



\vspace{3em}
\section{Measure Spaces}

\begin{definition}
A \textit{(positive) measure} \(\mu\) on \(X\) is a map \(\mu : \mathcal{A} \to [0, \infty]\), where \(\mathcal{A}\) is a \(\sigma\)-algebra on \(X\), satisfying:
\[
\mu(\emptyset) = 0, \tag{M1}
\]
and for any pairwise disjoint sequence \((A_n)_{n \in \mathbb{N}} \subseteq \mathcal{A}\),
\[
\mu\left( \bigsqcup_{n \in \mathbb{N}} A_n \right) = \sum_{n \in \mathbb{N}} \mu(A_n). \tag{M2}
\]
Property (M2) is also called \emph{countable additivity}.

If \(\mu\) satisfies (M1), (M2), but \(\mathcal{A}\) is not a \(\sigma\)-algebra, then \(\mu\) is called a \textit{premeasure}.
\end{definition}

\medskip
\begin{remark}
(M2) requires implicitly that \(\bigsqcup_{n} A_n\) is again in \(\mathcal{A}\) this is clearly the case for \(\sigma\)-algebras, but needs special attention when dealing with pre-measures.
\end{remark}


\medskip
\begin{definition}[Monotone sequences of sets]
Let \((A_n)_{n \in \mathbb{N}}\) and \((B_n)_{n \in \mathbb{N}}\) be sequences of subsets of \(X\).

We say \((A_n)\) is \emph{increasing} if 
\[
A_1 \subseteq A_2 \subseteq A_3 \subseteq \cdots
\]
and write \(A_n \uparrow A\) where
\[
A := \bigcup_{n \in \mathbb{N}} A_n
\]

Similarly, \((B_n)\) is \emph{decreasing} if
\[
B_1 \supseteq B_2 \supseteq B_3 \supseteq \cdots
\]
and write \(B_n \downarrow B\) where
\[
B := \bigcap_{n \in \mathbb{N}} B_n
\]
\end{definition}


\medskip
\begin{definition}
Let \(X\) be a set and \(\mathcal{A}\) a \(\sigma\)-algebra on \(X\). The pair \((X, \mathcal{A})\) is called a \emph{measurable space}.  
If \(\mu\) is a measure on \((X, \mathcal{A})\), then \((X, \mathcal{A}, \mu)\) is called a \emph{measure space}.

\medskip

A measure \(\mu\) is called:
\begin{itemize}
  \item \emph{finite} if \(\mu(X) < \infty\),
  \item a \emph{probability measure} if \(\mu(X) = 1\).
\end{itemize}

Accordingly, we speak of a \emph{finite measure space} and a \emph{probability space}.
\end{definition}


\medskip
\begin{definition}
A measure \(\mu\) on a measurable space \((X, \mathcal{A})\) is called \(\sigma\)-\emph{finite} if there exists a sequence \((A_n)_{n \in \mathbb{N}} \subseteq \mathcal{A}\) such that:
\[
A_n \uparrow X \quad \text{and} \quad \mu(A_n) < \infty \quad \text{for all } n \in \mathbb{N}.
\]
In this case, the measure space \((X, \mathcal{A}, \mu)\) is called \emph{\(\sigma\)-finite}.
\end{definition}


\medskip
\begin{lemma}[Basic properties of measures]
Let \((X,\mathcal{A},\mu)\) be a measure space. Then:
\renewcommand{\labelenumi}{(\roman{enumi})}
\begin{enumerate}
    \item If \(A_0, \ldots, A_k \in \mathcal{A}\) are pairwise disjoint, then \(\mu\big(\bigcup_{n=0}^k A_n\big) = \sum_{n=0}^k \mu(A_n)\).
    \item If \(A, B \in \mathcal{A}\) with \(A \subseteq B\), then \(\mu(A) \leq \mu(B)\).
    \item If \(A, B \in \mathcal{A}\), \(A \subseteq B\), and \(\mu(A) < \infty\), then \(\mu(B \setminus A) = \mu(B) - \mu(A)\).
\end{enumerate}
\end{lemma}

\begin{proof}
(i) Extend \((A_n)\) by \(A_n = \varnothing\) for \(n > k\). Then by countable additivity,
\[
\mu\big(\bigcup_{n=1}^k A_n\big) = \mu\big(\bigcup_{n=1}^\infty A_n\big) = \sum_{n=1}^\infty \mu(A_n) = \sum_{n=1}^k \mu(A_n).
\]

(ii) Since \(B = A \cup (B \setminus A)\) with disjoint union,
\[
\mu(B) = \mu(A) + \mu(B \setminus A) \geq \mu(A).
\]

(iii) Rearranging gives
\[
\mu(B \setminus A) = \mu(B) - \mu(A),
\]
which is well-defined if \(\mu(A) < \infty\).
\end{proof}


\medskip
\begin{lemma}[Main properties of measures]
Let \((X, \mathcal{A}, \mu)\) be a measure space. Then:
\medskip
\begin{enumerate}
    \item[(i)] \textbf{Countable subadditivity:} For any countable family \(\{A_i\}_{i \in \mathbb{N}} \subseteq \mathcal{A}\),
    \[
    \mu\left( \bigcup_{i \in \mathbb{N}} A_i \right) \leq \sum_{i \in \mathbb{N}} \mu(A_i).
    \]

    \item[(ii)] \textbf{Continuity from below (increasing sequence):} If \(A_1 \subseteq A_2 \subseteq \cdots\) (i.e., \(A_n \uparrow A\)), then
    \[
    \mu\left( \bigcup_{n \in \mathbb{N}} A_n \right) = \lim_{n \to \infty} \mu(A_n).
    \]

    \item[(iii)] \textbf{Continuity from above (decreasing sequence):} If \(B_1 \supseteq B_2 \supseteq \cdots\) (i.e., \(B_n \downarrow B\)), then
    \[
    \mu\left( \bigcap_{n \in \mathbb{N}} B_n \right) = \lim_{n \to \infty} \mu(B_n).
    \]
\end{enumerate}
\end{lemma}

\medskip
\begin{proof}
(i) For countable subadditivity, set \(B_k := A_k \setminus \bigcup_{i=1}^{k-1} A_i\), so that \((B_k)\) are disjoint with \(B_k \subseteq A_k\). Then,
\[
\mu\left(\bigcup_{i=1}^\infty A_i\right) = \mu\left(\bigsqcup_{i=1}^\infty B_i\right) = \sum_{i=1}^\infty \mu(B_i) \leq \sum_{i=1}^\infty \mu(A_i).
\]

\vspace{3em}
(ii) Let \( A_n \uparrow A \), i.e., \( A_n \subseteq A_{n+1} \) and \( A = \bigcup_{n \in \mathbb{N}} A_n \). Define \( B_n := A_n \setminus A_{n-1} \) with \( A_0 := \emptyset \). Then \( (B_n) \) is disjoint and \( \bigsqcup_{n} B_n = A \). By countable additivity,
\[
\mu(A) = \sum_{n=1}^\infty \mu(B_n) = \lim_{n \to \infty} \sum_{k=1}^n \mu(B_k) = \lim_{n \to \infty} \mu(A_n) 
\]

\vspace{3em}
(iii) Assume \(B_n \downarrow B\), i.e., \(B_n \supseteq B_{n+1}\) and \(B = \bigcap_{n} B_n\), with \(\mu(B_1) < \infty\). Set \(A_n := B_1 \setminus B_n\), so \(A_n \uparrow A := B_1 \setminus B\). Then
\[
\mu(B) = \mu(B_1) - \mu(A) = \mu(B_1) - \lim_{n \to \infty} \mu(A_n) = \lim_{n \to \infty} \mu(B_n)
\]
\end{proof}

\begin{remark}
With appropriate modifications, these properties also hold for pre-measures, i.e., when \(\mathcal{A}\) is not necessarily a \(\sigma\)-algebra.
\end{remark}


\medskip
\begin{example}[Dirac measure]
Let \((X, \mathcal{A})\) be a measurable space and let \(x \in X\). Define \(\delta_x : \mathcal{A} \to \{0,1\}\) by
\[
\delta_x(A) := 
\begin{cases}
1 & \text{if } x \in A, \\
0 & \text{if } x \notin A.
\end{cases}
\]
Then \(\delta_x\) is a measure on \((X, \mathcal{A})\), called the \emph{Dirac measure} (or unit mass) at the point \(x\).
\end{example}

\medskip
\begin{example}[Counting measure]
Let \((X, \mathcal{A})\) be a measurable space. Define \(\#A : \mathcal{A} \to [0, \infty]\) by
\[
\#A :=
\begin{cases}
\text{number of elements in } A & \text{if } A \text{ is finite}, \\
\infty & \text{if } A \text{ is infinite}.
\end{cases}
\]
Then \(\#\) is a measure on \((X, \mathcal{A})\), called the \emph{counting measure}.
\end{example}


\medskip
\begin{example}[Discrete probability measure]
Let \(\Omega = \{ \omega_1, \omega_2, \dots \}\) be a countable set, and let \((p_n)_{n \in \mathbb{N}} \subseteq [0,1]\) be a sequence such that \(\sum_{n \in \mathbb{N}} p_n = 1\). Define the set function \(P : \mathcal{P}(\Omega) \to [0,1]\) by
\[
P(A) := \sum_{\{n \in \mathbb{N} : \omega_n \in A\}} p_n = \sum_{n \in \mathbb{N}} p_n \, \delta_{\omega_n}(A), \quad A \subseteq \Omega,
\]
where \(\delta_{\omega_n}\) denotes the Dirac measure at \(\omega_n\). Then \(P\) is a probability measure on \((\Omega, \mathcal{P}(\Omega))\), and the triplet \((\Omega, \mathcal{P}(\Omega), P)\) is called a \emph{discrete probability space}.
\end{example}


\medskip
\begin{example}[Linear combination of measures]
Let \((X, \mathcal{A})\) be a measurable space, and let \((\mu_n)_{n \in \mathbb{N}}\) be a sequence of measures on \((X, \mathcal{A})\). Let \((x_n)_{n \in \mathbb{N}} \subseteq [0, \infty]\). Then the set function
\[
\mu := \sum_{n \in \mathbb{N}} x_n \mu_n
\]
defined by
\[
\mu(A) := \sum_{n \in \mathbb{N}} x_n \mu_n(A), \quad \text{for all } A \in \mathcal{A},
\]
is a measure on \((X, \mathcal{A})\)
\end{example}

\begin{proof}
We verify the axioms of a measure:

\textbf{(M1)} \emph{(Null empty set):} For all \( n \in \mathbb{N} \), \( \mu_n(\emptyset) = 0 \), so
\[
\mu(\emptyset) = \sum_{n \in \mathbb{N}} x_n \mu_n(\emptyset) = \sum_{n \in \mathbb{N}} x_n \cdot 0 = 0.
\]

\textbf{(M2)} \emph{(Countable additivity):} Let \( (A_k)_{k \in \mathbb{N}} \subseteq \mathcal{A} \) be pairwise disjoint. Since each \( \mu_n \) is a measure, we have
\[
\mu_n\left( \bigsqcup_{k \in \mathbb{N}} A_k \right) = \sum_{k \in \mathbb{N}} \mu_n(A_k), \quad \text{for all } n \in \mathbb{N}.
\]
Then,
\[
\mu\left( \bigsqcup_{k \in \mathbb{N}} A_k \right)
= \sum_{n \in \mathbb{N}} x_n \mu_n\left( \bigsqcup_{k \in \mathbb{N}} A_k \right)
= \sum_{n \in \mathbb{N}} x_n \sum_{k \in \mathbb{N}} \mu_n(A_k).
\]
Since all terms are non-negative, we may exchange the order of summation:
\[
\sum_{n \in \mathbb{N}} x_n \sum_{k \in \mathbb{N}} \mu_n(A_k)
= \sum_{k \in \mathbb{N}} \sum_{n \in \mathbb{N}} x_n \mu_n(A_k)
= \sum_{k \in \mathbb{N}} \mu(A_k).
\]
Therefore, \( \mu \) is countably additive.
\end{proof}


\medskip
\begin{example}[Restriction of a measure]
Let \((X, \mathcal{A}, \mu)\) be a measure space and let \(A \in \mathcal{A}\). Define the set function \(\mu_{\!A} : \mathcal{A} \to [0, \infty]\) by
\[
\mu_{\!A}(B) := \mu(A \cap B), \quad \text{for all } B \in \mathcal{A}.
\]
Then \(\mu_{\!A}\) is a measure on \((X, \mathcal{A})\), called the \emph{restriction of \(\mu\) to \(A\)}.
\end{example}

\begin{proof}
We verify the two defining properties of a measure:

\textbf{(M1)}: \(\mu_{\!A}(\emptyset) = \mu(A \cap \emptyset) = \mu(\emptyset) = 0\).

\medskip
\textbf{(M2)}: Let \((B_n)_{n \in \mathbb{N}} \subseteq \mathcal{A}\) be pairwise disjoint. Then \((A \cap B_n)_{n \in \mathbb{N}}\) are also pairwise disjoint, and
\[
\mu_{\!A}\left( \bigsqcup_{n \in \mathbb{N}} B_n \right)
= \mu\left( A \cap \bigsqcup_{n \in \mathbb{N}} B_n \right)
= \mu\left( \bigsqcup_{n \in \mathbb{N}} (A \cap B_n) \right)
= \sum_{n \in \mathbb{N}} \mu(A \cap B_n)
= \sum_{n \in \mathbb{N}} \mu_{\!A}(B_n).
\]
Hence, \(\mu_{\!A}\) is a measure.
\end{proof}

\medskip
\begin{definition}[Lebesgue measure on \(\mathbb{R}^n\)]
Define the set function \(\lambda_n\) on \((\mathbb{R}^n, \mathcal{B}(\mathbb{R}^n))\) by
\[
\lambda_n\left( \llbracket a, b \rrparenthesis \right) := \prod_{i=1}^n (b_i - a_i),
\]
for all \(\llbracket a, b \rrparenthesis := [a_1, b_1) \times \cdots \times [a_n, b_n) \in \mathcal{J}_n\).  
This is called the \(n\)-dimensional Lebesgue measure.
\end{definition}

\begin{remark}
The set function \(\lambda_n\) is defined only on the family \(\mathcal{J}_n\) of half-open rectangles and hence is not yet a measure. Extending \(\lambda_n\) to a measure on \(\mathcal{B}(\mathbb{R}^n)\) requires the Carathéodory extension theorem, which will be developed later.
\end{remark}


\medskip
\begin{theorem}[Existence and properties of Lebesgue measure]
There exists a unique measure \(\lambda_n\) on \((\mathbb{R}^n, \mathcal{B}(\mathbb{R}^n))\) extending the pre-measure defined on \(\mathcal{J}_n\). Moreover, for all \(B \in \mathcal{B}(\mathbb{R}^n)\), \(\lambda_n\) satisfies:
\begin{itemize}
    \item[(i)] \textbf{Translation invariance:} \(\lambda_n(x + B) = \lambda_n(B)\) for all \(x \in \mathbb{R}^n\).
    \item[(ii)] \textbf{Motion invariance:} \(\lambda_n(R^{-1}(B)) = \lambda_n(B)\) for any motion \(R\), i.e., composition of translations, rotations, and reflections.
    \item[(iii)] \textbf{Linear change of variables:} \(\lambda_n(M^{-1}(B)) = |\det(M)|^{-1} \lambda_n(B)\) for any invertible matrix \(M \in \mathbb{R}^{n \times n}\).
\end{itemize}
These properties will be established later, once the necessary tools have been developed.
\end{theorem}


\medskip
\begin{lemma}
Let \((X, \mathcal{A})\) be a measure space, and let \(\mu : \mathcal{A} \to [0, \infty]\) be an additive set function with \(\mu(\emptyset) = 0\). Then \(\mu\) is a measure if and only if it is \textbf{continuous from below}, i.e., for every increasing sequence \((A_n)_{n \in \mathbb{N}} \subseteq \mathcal{A}\) with \(A_n \uparrow A\), we have
\[
\mu(A) = \lim_{n \to \infty} \mu(A_n) = \sup_{n \in \mathbb{N}} \mu(A_n).
\]
\end{lemma}

\begin{proof}
Any measure \(\mu\) is continuous from below.

Conversely, suppose \(\mu\) is finitely additive, \(\mu(\emptyset) = 0\), and \(\mu\) is continuous from below. Let \((B_n)_{n \in \mathbb{N}} \subseteq \mathcal{A}\) be disjoint, and define \(A_n := \bigcup_{i=1}^n B_i\). Then \((A_n)\) is increasing with \(\bigcup_{n=1}^\infty A_n = \bigsqcup_{n=1}^\infty B_n\). By finite additivity,
\[
\mu(A_n) = \sum_{i=1}^n \mu(B_i),
\]
and by continuity from below,
\[
\mu\left(\bigsqcup_{n=1}^\infty B_n\right) = \mu\left(\bigcup_{n=1}^\infty A_n\right) = \lim_{n \to \infty} \mu(A_n) = \sum_{n=1}^\infty \mu(B_n).
\]
Hence \(\mu\) is countably additive, i.e., a measure.
\end{proof}


\medskip
\begin{lemma}
Let \((X, \mathcal{A})\) be a measurable space and \(\mu : \mathcal{A} \to [0, \infty)\) an additive set function with \(\mu(\emptyset) = 0\) and \(\mu(A) < \infty\) for all \(A \in \mathcal{A}\). Then \(\mu\) is a measure if and only if it satisfies one of the following continuity properties:

\begin{enumerate}
  \item[(i)] \(\mu\) is continuous from below;
  \item[(ii)] \(\mu\) is continuous from above;
  \item[(iii)] \(\mu\) is continuous at \(\emptyset\), i.e., for every decreasing sequence \((B_n)_{n \in \mathbb{N}}\) in \(\mathcal{A}\) with \(\bigcap_{n=0}^\infty B_n = \emptyset\), we have
  \[
    \lim_{n \to \infty} \mu(B_n) = 0.
  \]
\end{enumerate}
\end{lemma}

\begin{proof}
Clearly, every measure satisfies properties (i)–(iii), so we only need to show that (iii) implies countable additivity.

Assume \(\mu\) is additive, \(\mu(\emptyset) = 0\), and satisfies continuity at \(\emptyset\). Let \((A_n)_{n \in \mathbb{N}} \subseteq \mathcal{A}\) be pairwise disjoint and define \(A := \bigsqcup_{n \in \mathbb{N}} A_n\). For each \(n\), let
\[
B_n := A \setminus \bigcup_{i=1}^n A_i.
\]
Then \((B_n)\) is a decreasing sequence in \(\mathcal{A}\) with \(\bigcap_{n \in \mathbb{N}} B_n = \emptyset\), so by continuity at \(\emptyset\), we have \(\mu(B_n) \to 0\).

Using additivity, we compute
\[
\mu(A) = \mu\left(B_n \sqcup \bigcup_{i=1}^n A_i\right) = \mu(B_n) + \sum_{i=1}^n \mu(A_i).
\]
Taking the limit as \(n \to \infty\), we get
\[
\mu(A) = \lim_{n \to \infty} \left( \mu(B_n) + \sum_{i=1}^n \mu(A_i) \right) = \sum_{i=1}^\infty \mu(A_i).
\]
Thus, \(\mu\) is countably additive, hence a measure.
\end{proof}



\vspace{3em}
\section{Uniqueness}

\begin{definition}
A \emph{Dynkin system} (or \(\lambda\)-system) \(\mathcal{D} \subseteq \mathcal{P}(X)\) is a collection of subsets of \(X\) such that:
\begin{enumerate}
    \item \(X \in \mathcal{D} \hfill \text{(\textbf{D1})}\)
    \item If \(D \in \mathcal{D}\), then \(D^c \in \mathcal{D} \hfill \text{(\textbf{D2})}\)
    \item If \((D_n)_{n \in \mathbb{N}} \subseteq \mathcal{D}\) are pairwise disjoint, then \(\bigsqcup_{n \in \mathbb{N}} D_n \in \mathcal{D} \hfill \text{(\textbf{D3})}\)
\end{enumerate}
\end{definition}


\end{document}