\documentclass{article}
\usepackage{amsmath, amsthm, amssymb, amsfonts}
\usepackage{hyperref}

\title{\textbf{Measure Theory}\\[0.5em]}
\author{}
\date{}

\theoremstyle{definition}
\newtheorem{definition}{Definition}[section]
\newtheorem{theorem}{Theorem}[section]
\newtheorem{corollary}{Corollary}[theorem]
\newtheorem{lemma}[theorem]{Lemma}
\newtheorem{example}{Example}[section]
\newtheorem{remark}{Remark}[section]


\begin{document}
\maketitle

\section{$\sigma$-Algebras}
\begin{definition}
A $\sigma$-algebra $\mathcal{A}$ on a set $X$ is a family of subsets of $X$ such that:
\begin{enumerate}
    \item $X \in \mathcal{A}$,
    \item If $A \in \mathcal{A}$, then $A^c \in \mathcal{A}$,
    \item If $(A_n)_{n \in \mathbb{N}} \subseteq \mathcal{A}$, then $\bigcup_{n \in \mathbb{N}} A_n \in \mathcal{A}$.
\end{enumerate}

\medskip

A set $A \in \mathcal{A}$ is said to be \textit{measurable} or \textit{$\mathcal{A}$-measurable}.
\end{definition}


\medskip
\begin{example}\leavevmode\
\begin{enumerate}
    \item $\mathcal{P}(X)$ is a $\sigma$-algebra (the maximal $\sigma$-algebra on $X$).
    
    \item $\{\emptyset, X\}$ is a $\sigma$-algebra (the minimal $\sigma$-algebra on $X$).
    
    \item $A := \{A \subseteq X : \#A < \infty \text{ or } \#A^c < \infty\}$ is a $\sigma$-algebra.

    \item (Trace $\sigma$-algebra) Let $E \subseteq X$ be any set and let $\mathcal{A}$ be a $\sigma$-algebra on $X$. Then
    \[
    \mathcal{A}_E := \{E \cap A : A \in \mathcal{A}\}
    \]
    is a $\sigma$-algebra on $E$.

    \begin{proof}
We verify the three defining properties of a $\sigma$-algebra on $E$:

\begin{enumerate}
    \item Since $X \in \mathcal{A}$, we have
    \[
    E = E \cap X \in \mathcal{A}_E.
    \]
    \item If $E \cap A \in \mathcal{A}_E$, then
    \[
    E \setminus (E \cap A) = E \cap A^c,
    \]
    and since $A^c \in \mathcal{A}$, it follows that
    \[
    E \cap A^c \in \mathcal{A}_E.
    \]
    \item If $(E \cap A_n)_{n \in \mathbb{N}}$ is a sequence in $\mathcal{A}_E$, then
    \[
    \bigcup_{n} (E \cap A_n) = E \cap \bigcup_{n} A_n,
    \]
    and since $\bigcup_n A_n \in \mathcal{A}$, we conclude that
    \[
    \bigcup_n (E \cap A_n) \in \mathcal{A}_E.
    \]
\end{enumerate}
Hence, $\mathcal{A}_E$ is a $\sigma$-algebra on $E$.
\end{proof}

    \item (Pre-image $\sigma$-algebra) Let $f : X \to X'$ be a function and let $\mathcal{A}'$ be a $\sigma$-algebra on $X'$. Then
    \[
    \mathcal{A} := \{f^{-1}(A') : A' \in \mathcal{A}'\}
    \]
    is a $\sigma$-algebra on $X$.
\end{enumerate}
\end{example}


\medskip
\begin{theorem}
Let $X$ be a set and let $\{\mathcal{A}_i : i \in I\}$ be a family of $\sigma$-algebras on $X$. Define
\[
\mathcal{A} := \bigcap_{i \in I} \mathcal{A}_i = \{ A \subseteq X : A \in \mathcal{A}_i \text{ for all } i \in I \}.
\]
Then, $\mathcal{A}$ is a $\sigma$-algebra on $X$.
\end{theorem}

\begin{proof}
We verify the $\sigma$-algebra properties for $\mathcal{A}$:
\begin{enumerate}
    \item Since $X \in \mathcal{A}_i$ for all $i \in I$, we have $X \in \mathcal{A}$.
    \item If $A \in \mathcal{A}$, then $A \in \mathcal{A}_i$ for all $i \in I$, so
    \[
    A^c = X \setminus A \in \mathcal{A}_i \quad \forall i \in I,
    \]
    hence $A^c \in \mathcal{A}$.
    \item If $(A_n)_{n \in \mathbb{N}} \subseteq \mathcal{A}$, then for all $n$ and $i$,
    \[
    A_n \in \mathcal{A}_i,
    \]
    so by closure of each $\mathcal{A}_i$ under countable unions,
    \[
    \bigcup_{n=0}^\infty A_n \in \mathcal{A}_i \quad \forall i \in I,
    \]
    and thus
    \[
    \bigcup_{n=0}^\infty A_n \in \mathcal{A}.
    \]
\end{enumerate}
Therefore, $\mathcal{A}$ is a $\sigma$-algebra on $X$.
\end{proof}


\medskip
\begin{definition}
Let $X$ be a set and let $\mathcal{E} \subseteq \mathcal{P}(X)$ be a collection of subsets of $X$. The \textit{$\sigma$-algebra generated by} $\mathcal{E}$, denoted by $\sigma(\mathcal{E})$, is the smallest $\sigma$-algebra on $X$ containing all sets in $\mathcal{E}$. That is,

\[
\sigma(\mathcal{E}) := \bigcap \big\{ \mathcal{A} \subseteq \mathcal{P}(X) :\, \mathcal{A} \text{ is a } \sigma\text{-algebra on } X, \; \mathcal{E} \subseteq \mathcal{A} \big\}.
\]
\end{definition}


\medskip
\begin{remark}[Generated $\sigma$-algebras]
\leavevmode
\begin{enumerate}
    \item If $\mathcal{A}$ is a $\sigma$-algebra, then $\sigma(\mathcal{A}) = \mathcal{A}$.
    \item For $A \subseteq X$, we have $\sigma(\{A\}) = \{\emptyset, A, A^c, X\}$.
    \item If $\mathcal{F} \subseteq \mathcal{G} \subseteq \mathcal{A}$, then $\sigma(\mathcal{F}) \subseteq \sigma(\mathcal{G}) \subseteq \sigma(\mathcal{A})$.
\end{enumerate}
\end{remark}

\medskip
\begin{definition}[Topological Space]
A \textit{topological space} is a pair $(X, \mathcal{T})$ where $X$ is a set and $\mathcal{T} \subseteq \mathcal{P}(X)$ is a collection of subsets of $X$, called \textit{open sets}, satisfying the following properties:
\begin{enumerate}
    \item $\emptyset \in \mathcal{T}$ and $X \in \mathcal{T}$,
    \item If $\{U_\alpha \in \mathcal{T} : \alpha \in I\}$ is an arbitrary collection of open sets, then the union $\bigcup_{\alpha \in I} U_\alpha \in \mathcal{T}$,
    \item If $\{U_i \in \mathcal{T} : i = 1, \dots, N\}$ is a finite collection of open sets, then the intersection $\bigcap_{i=1}^N U_i \in \mathcal{T}$.
\end{enumerate}
The collection $\mathcal{T}$ is called a \textit{topology} on $X$. The complement of an open set is called a \textit{closed set}.
\end{definition}


\medskip
\begin{remark}[Standard Topology on \(\mathbb{R}^n\)]
A subset \( U \subseteq \mathbb{R}^n \) is called \textit{open} if for every point \( x \in U \), there exists an \(\varepsilon > 0\) such that the open ball
\[
B_\varepsilon(x) := \{ y \in \mathbb{R}^n : \|x - y\| < \varepsilon \},
\]
where \(\|\cdot\|\) denotes the Euclidean norm, is contained in \( U \); that is, \( B_\varepsilon(x) \subseteq U \).

The collection of all such open sets is denoted by \(\mathcal{O} = \mathcal{O}_{\mathbb{R}^n}\) and forms the \textit{standard topology} on \(\mathbb{R}^n\).
\end{remark}


\medskip
\begin{definition}[Borel $\sigma$-algebra]
The \(\sigma\)-algebra \(\sigma(\mathcal{O})\) generated by the collection of open sets \(\mathcal{O} = \mathcal{O}_{\mathbb{R}^n}\) of \(\mathbb{R}^n\) is called the \textit{Borel \(\sigma\)-algebra} on \(\mathbb{R}^n\). 

Its elements are called \textit{Borel sets} or \textit{Borel measurable sets}. We denote the Borel \(\sigma\)-algebra by \(\mathcal{B}(\mathbb{R}^n)\).
\end{definition}
\end{document}